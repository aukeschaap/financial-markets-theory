\documentclass[twoside]{article}

\usepackage[utf8]{inputenc}
\usepackage[T1]{fontenc}

\usepackage{multicol}
\usepackage{enumerate}
\usepackage{graphicx}
\usepackage[font=small,labelfont=bf]{caption}
\usepackage[dvipsnames]{xcolor}

\usepackage[a4paper]{geometry}
\geometry{
  a4paper,
  left=25mm,
  right=25mm,
  top=30mm,
  bottom=30mm
}



% Better fonts
% \usepackage{newpxtext}
\usepackage{cmbright}
\usepackage{lmodern}

% Removes indentation and adds vspace
%   NOTE: add 1em below each theorem if disabled
\usepackage{parskip}

% Equations and mathematical symbols
\usepackage{amsmath, amssymb, amsthm, mathtools}
\usepackage{calrsfs}

% Hyperlinks
\usepackage{hyperref}
\usepackage{cleveref}

% Appendices
\usepackage{appendix}


% Code listings
\usepackage{listings}
\usepackage{transparent}
\lstset{frame=tb,
  language=Python,
  aboveskip=7mm,
  belowskip=7mm,
  columns=flexible,
  basicstyle={\footnotesize\ttfamily},
  keywordstyle={\bfseries\color{NavyBlue}},
  commentstyle=\color{Gray},
  stringstyle=\color{Green},
  breaklines=false,
  breakatwhitespace=true,
  tabsize=3,
  morecomment=[l][\color{Gray}\transparent{0.5}]{\#},
}


% Theorems
\usepackage{tikz}
\usepackage{tikz-cd}
\usepackage{thmtools}
\usepackage[framemethod=TikZ]{mdframed}
\mdfsetup{skipabove=1em, skipbelow=0em, innertopmargin=5pt, innerbottommargin=6pt} 

\theoremstyle{definition}

\declaretheoremstyle[headfont=\bfseries\sffamily, bodyfont=\normalfont, mdframed={ nobreak,}]{thmbox}
\declaretheoremstyle[headfont=\bfseries\sffamily, bodyfont=\normalfont, numbered=no, mdframed={ nobreak, rightline=false, topline=false, bottomline=false}]{infobox}

\declaretheorem[style=thmbox, name=Definition]{definition}
\declaretheorem[sibling=definition, style=thmbox, name=Corollary]{corollary}
\declaretheorem[sibling=definition, style=thmbox, name=Proposition]{proposition}
\declaretheorem[sibling=definition, style=thmbox, name=Theorem]{theorem}
\declaretheorem[sibling=definition, style=thmbox, name=Lemma]{lemma}

\declaretheorem[style=infobox, name=Assumption]{assumption}
\declaretheorem[style=infobox, name=Explanation]{explanation}
\declaretheorem[style=infobox, name=Exercise]{ex}
\declaretheorem[style=infobox, name=Example]{eg}
\declaretheorem[style=infobox, name=Remark]{remark}
\declaretheorem[style=infobox, name=Note]{note}

\declaretheoremstyle[headfont=\bfseries\sffamily, bodyfont=\normalfont, numbered=no, mdframed={} ]{thmsolutionbox}
\declaretheorem[numbered=no, style=thmsolutionbox, name=Solution]{solution}


\usepackage{import}
\usepackage{xifthen}
\usepackage{pdfpages}
\usepackage{transparent}

\newcommand{\incfig}[1]{%
    \def\svgwidth{0.8\columnwidth}
    \import{./figures/}{#1.pdf_tex}
}


\makeatletter

% Set space around theorems
\def\thm@space@setup{%
  \thm@preskip=.5\topsep \thm@postskip=0pt
}

\def\@lecture{}%
\newcommand{\lecture}[1]{
    \def\@lecture{Lecture #1}
}

% These are the fancy headers
\usepackage{fancyhdr}
\pagestyle{fancy}
\setlength{\headheight}{13pt}

% LE: left even
% RO: right odd
% CE, CO: center even, center odd
% My name for when I print my lecture notes to use for an open book exam.
% \fancyhead[LE,RO]{Auke Schaap}

\fancyhead[RO,LE]{\@lecture} % Right odd,  Left even
\fancyhead[RE,LO]{}          % Right even, Left odd

\fancyfoot[RO,LE]{\thepage}  % Right odd,  Left even
\fancyfoot[RE,LO]{}          % Right even, Left odd
\fancyfoot[C]{\leftmark}     % Center

\makeatother



%%%%%%%%%%%%%%%%%
% Custom macros %
%%%%%%%%%%%%%%%%%

\newcommand\N{\ensuremath{\mathbb{N}}}
\newcommand\R{\ensuremath{\mathbb{R}}}
\newcommand\E{\ensuremath{\mathbb{E}}}


% (*) above a symbol
\newcommand\mygt{\stackrel{\mathclap{\scriptsize\mbox{$(\ast)$}}}{>}}

\newif\ifshowcomments
\showcommentstrue
% \showcommentsfalse
\ifshowcomments
\newcommand{\mynote}[2]{\fbox{\bfseries\sffamily\scriptsize{#1}}
{\small$\blacktriangleright$\textsf{\emph{#2}}$\blacktriangleleft$}}
\newcommand{\citehere}[0]{\textcolor{red}{\fbox{\bfseries\sffamily\scriptsize{CITATION}}}}
\else
\newcommand{\mynote}[2]{}
\fi
\newcommand{\todo}[1]{\textcolor{blue}{\mynote{To do}{#1}}}

\title{%
    Summary \\
    \large WI4425 - Financial Markets Theory
}
\author{Auke Schaap}
\date{June 2023}

\begin{document}
\maketitle
\newpage


\begin{table}[h]
    \centering
    \begin{tabular}{l|llll}
    Lecture &  Reading  &       Exercises & Document \\
    \hline
    1 &        1.1, 1.2 &       1 \& 2 & FMT 1 \\
    2 &        1.3, 2.1 &       3 \& 5 & FMT 1 \\
    3 &        2.1, 2.2 &       4, 1 & FMT 1, FMT 2 \\
    4 &        2.2, 2.3 &       2 & FMT 2 \\
    5 &        2.3 &            3.6 \& 3.8 & Book \\
    6 &        2.3, 2.4 &       1 & Insurance Dominance \\
    7 &        3.1, 3.2 &       2 & Insurance Dominance \\
    8 &        3.3, 3.4, 5.1 &  All & Portfolio \\
    9 &        5.2, 5.3 &       All & Risk neutral measure \\
    10&        4.4 &            All & Risk neutral measure
    \end{tabular}
    \caption{Overview of the lectures, readings, exercises and documents.}
\end{table}

\begin{table}[h]
    \centering
    \begin{tabular}{l|l}
    Chapter & Sections \\
    \hline
    1 & 1.1, 1.2, 1.3 \\
    2 & 2.1, 2.2, 2.3, 2.4 \\
    3 & 3.1, 3.2, 3.3, 3.4 \\
    4 & 4.4 \\
    5 & 5.1, 5.2, 5.3
    \end{tabular}
    \caption{Overview of the chapters and sections.}
\end{table}

\newpage

\lecture{1}
\section{Choices under certainty}

The couple $(X, \mathcal{R})$, represents the \textit{agent's choice problem}, where
\begin{itemize}
    \item $X$ is a set of choices, which, in the classical consumption problem, concerns bundles of goods. $X$ is therefore a subset of $\R^L$, where $L \in \N$ is the number of goods.
    \item $\mathcal{R}$ is a (weak) \textit{preference relation} on $X$. Given $x,y \in X$, if $x \mathcal{R} y$ we say that the bundle of goods $x$ is at least as preferred as the bundle of goods $y$.
\end{itemize}

From this we can derive a two other preference relations. The \textit{strong preference relation} $\mathcal{P}$ is defined as $x \mathcal{P} y \iff x \mathcal{R} y$ \text{but not} $y \mathcal{R} x$. The \textit{indifference relation} $\mathcal{I}$ is defined as $x \mathcal{I} y \iff x \mathcal{R} y $ and $y \mathcal{R} x$.

\begin{assumption}[Rationality] \label{ass:rationality}
    The preference relation $\mathcal{R}$ is said to be \textit{rational} if it satisfies:
    \begin{itemize}
        \item \textit{Reflexivity}: $x \mathcal{R} x$ for all $x \in X$.
        \item \textit{Completeness}: For all $x, y \in X$, either $x \mathcal{R} y$ or $y \mathcal{R} x$.
        \begin{explanation}
            This means that the agent is able to compare any two bundles of goods.
        \end{explanation}
        \item \textit{Transitivity}: For all $x, y, z \in X$, if $x \mathcal{R} y$ and $y \mathcal{R} z$, then $x \mathcal{R} z$.
        \begin{explanation}
            This means that the agent is able to rank all bundles of goods.
        \end{explanation}
    \end{itemize}
\end{assumption}

\begin{assumption}[Continuity] \label{ass:continuity}
    The preference relation $\mathcal{R}$ is said to be \textit{continuous} if for all $x \in X$, the sets $\{y \in X : y \mathcal{R} x\}$ and $\{y \in X : x \mathcal{R} y\}$ are closed.
\end{assumption}

Introducing a utility function $u : X \to \R$ that represents the preference relation $\mathcal{R}$ if 
\[
    x \mathcal{R} y \iff u(x) \geq u(y) \; \text{ for all } x, y \in X,
\]
we can write the agent's choice problem as the couple $(X, u)$. 

\begin{theorem}
    Let $\mathcal{R}$ be a rational and continuous preference relation on $X$. Then there exists a continuous utility function $u : X \to \R$ that represents $\mathcal{R}$.
\end{theorem}
Note that this does not say anything about the uniqueness of the utility function. If $u$ is a utility function that represents $\mathcal{R}$, then so does any strictly increasing transformation of $u$.

\begin{assumption}
    We furthermore assume that:
    \begin{itemize}
        \item $\mathcal{R}$ is strictly monotone if, for every $x, y \in X$ such that $y > x$, $x \mathcal{P} y$.
        \item $\mathcal{R}$ is strictly convex if, for every $x, y, z \in X$ with $x \not = y$ such that $x \mathcal{R} z$ and $y \mathcal{R} z$, for all $\alpha \in (0, 1)$, $\alpha x + (1 - \alpha) y \mathcal{P} z$.
    \end{itemize}
\end{assumption}
If we assume the utility functions to be twice continuously differentiable, then the first partial derivative of a utility function, the \textit{marginal utility}, is positive, and the second partial derivative is negative.

We assume that agents are characterized by \textit{substantial rationality}: An agent pursues his goals in the most appropriate way, under the constraints imposed by the environment. We, furthermore, dip the agent in a perfectly competitive market. This means that the agent is a \textit{price taker}. Given the wealth $w \in \R_{\geq0}$ and the price vector $p \in \R_{\geq 0}^L$, the set on which an agent then makes its choices (the \textit{budget constraint}) is
\[
B := \left\{x \in X : p^\top x \leq w\right\},
\]
for some closed set $X \subset \R_{\geq0}^L$. Given the couple $(X, u)$ and the budget constraint $B$, the \textit{optimal choice} is
\[
    x^* := \max_{x \in B} u(x).
\]
If the utility function is continuous, then there always exists a solution to this optimization problem, if $p \in \R_{>0}^L$.

\begin{explanation}
    As the utility function is considered to be strictly monotone, it is always optimal for an agent to spend all his wealth. Thus, the budget constraint becomes an equality constraint.
\end{explanation}

We can solve it by the Lagrangian method. The Lagrangian is:
\[
    \mathcal{L}(x, \lambda) = u(x) + \lambda (w - p^\top x).
\]
This means that we solve the set of equations:
\begin{align*}
    \frac{\partial u}{\partial x_l}\left(x^*\right) &= \lambda p_l, \; \forall l = 1, \dots, L, \\
    w &= p^\top x^*.
\end{align*}
We define the \textit{indirect utility function} as $v(\omega) \equiv u(x^*)$. 

Suppose we endow agent $i$ with endowment vector $e^i \in \R_{\geq0}^L$, and let $x(p)$ be the optimal consumption vector for price $p$. We can then write the demand $z^i(p)$ as
\[
    z^i(p) = x(p) - e^i.
\]

\begin{definition}[Internal consistency requirement]
    We require that agents optimize their utility functions under the budget constraint.
\end{definition}

\section{General Equilibrium Theory}
We have an economy with $I>1$ agents and $L>1$ goods. Every agent has solved his optimal consumption problem for his given wealth, $\omega^i = p^\top e^i$. The demand function summarizes his choices.

\begin{definition}[External consistency requirement]
    We solve the market by putting demand equal to supply:
    \[
        z(p) \equiv \sum^I_{i=1}(x^i(p) - e^i) = \mathbf{0}
    \]
\end{definition}
The price vector that solve this equation is the \textit{equilibrium price vector}. The couple $(p^*, x^*)$ is the \textit{equilibrium allocation}.

\begin{theorem}
    Let the preference relation $\mathcal{R}^i$ be rational, continuous, strictly monotone and strictly convex for all $i = 1, \dots, I$, and suppose that $\sum^I_{i=1}e^i \in \R_{>0}^L$. Then there exists a unique equilibrium price vector $p^* \in \R_{>0}^L$.
\end{theorem}

\lecture{2}

\section{Pareto optimality}
Given an economy described by $I$ couples $(\mathcal{R}^i, e^i)$, we define an allocation $x$ by:
\[
    x = (x^1, \dots, x^I) \;\; x^i \in \R_{\geq0}^L.
\]
This allocation is \textit{feasible} if we have
\[
    \sum^I_{i=1}x^i \leq \sum^I_{i=1}e^i.
\]
Given two feasible allocations $x$ and $y$, then $x$ \textit{Pareto dominates} $y$ if and only if
\begin{align*}
    x^i \mathcal{R} y^i \forall i = 1, \dots, I, \\
    \exists\; 1 \leq j \leq I : x^i \mathcal{P} y^i.
\end{align*}
\begin{explanation}
    $x$ is at least as preferred as $y$ for all agents, and strictly preferred by at least one agent. Thus, it is a better alternative.
\end{explanation}

A feasible allocation $x$ is an \textit{efficient allocation}, or \textit{Pareto optimal allocation}, if there is no other feasible allocation $y$ such that $y$ Pareto dominates $x$.

\begin{theorem}[First Welfare Theorem]
    If $(p^*, x^*)$ is a competitive equilibrium (and the preference relations of the agents are strictly monotone), then the allocation $x^*$ is Pareto optimal.
    \begin{explanation}
        So, there is an invisible hand in the market: agents maximizing their own utility in a perfectly competitive market lead to a socially optimal result.
    \end{explanation}
\end{theorem}

\begin{theorem}[Second Welfare Theorem]
    Let $x^*$ be a Pareto optimal allocation, $x^{i*} \in \R_{\geq0}^L$. If the preference relations $\mathcal{R}^i$ are rational, continuous, strictly monotone and strictly convex, then $x^*$ is the competitive equilibrium allocation, given that the initial endowments are $e^i = x^{i*}$.
    \begin{explanation}
        We can reach any Pareto optimal allocation by redistributing the goods in a proper way.
    \end{explanation}
\end{theorem}

Consider an economy $(u^i, e^i)$ and an associated competitive equilibrium $(p^*, x^*)$. Now consider an agent who has an optimal choice given the equilibrium price vector $p^*$ equal to the initial aggregate endowment. The agent has utility function
\begin{align*}
    \textbf{u}(x) := \max_{x \in \R_{\geq0}^L, i=1, \dots, I} \sum^I_{i=1}a^i u^i(x^i) \;\; \\ \text{s.t.} \sum^I_{i=1}x^i_l \leq \sum^I_{i=1}x_l.
\end{align*}
This is the \textit{social welfare function}. If the equilibrium price vector does not depend on the weights, we say that the economy satisfies the \textit{aggregation property}. If an economy satisfies the aggregation propery, then the economy with $I$ agents and the economy with the one representative agent are observationally equivalent.

\section{Choices Under Risk}

We now consider a state of the world at time $t=0$ and at $t=1$. The beliefs are modeled by a probability space $(\Omega, \mathcal{F}, \mu)$. We distinguish between:
\begin{itemize}
    \item \textit{Risk}: The probabilities are known.
    \item \textit{Uncertainty}: The probabilities are unknown.
\end{itemize}
Acts are modeled by random variables $\tilde{x}: \Omega \rightarrow \R$, where we assume $\tilde{x}$ is measurable in the following sense:
\[
    \forall B \in \mathcal{B}(\R) : \;\; \{ \omega \in \Omega : \tilde{x}(\omega) \in B \} \in \mathcal{F}.
\]
$\mu$ induces a probability measure $\pi$ on $\R$, $\pi(B) = \mu(\tilde{x}^{-1}(B))$, with $B \in \mathcal{B}(\R)$. The acts attain a finite set of values, specified a priori: $ [ x_1, \dots, x_S ], \; x_i < x_{i+1} \in \R, \;\; i = 1, \dots, S - 1$. We now have a set of acts, or \textit{gambles} $\mathcal{M}$, which are r.v.'s having support on the $x_i$ identified by a probability distribution:
\[
    \tilde{x} \in \mathcal{M} \iff \tilde{x} \equiv [x_1, \dots, x_S ; \pi_1, \dots, \pi_S] \;\; \text{s.t.} \;\; \pi_s \geq 0, \sum^S_{s=1}\pi_s = 1.
\]

\lecture{3}

\section{Expected Utility}
With the the characterization $(\mathcal{M}, \mathcal{R})$, we introduce a function of $\tilde{x}$ that represents $\mathcal{R}$, the \textit{expected utility}. We again assume rationality and continuity, but also independence:

\begin{assumption}[Independence]
    The preference relation $\mathcal{R}$ satisfies the \textit{independence} assumption, if for all $\tilde{x}_1, \tilde{x}_2, \tilde{x}_3 \in \mathcal{M}$ and $\alpha \in [0,1]$, we have
    \[
        \tilde{x}_1 \mathcal{R} \tilde{x}_2 \iff \alpha \tilde{x}_1 + (1 - \alpha)\tilde{x}_3 \mathcal{R} \alpha \tilde{x}_2 + (1 - \alpha)\tilde{x}_3.
    \]
\end{assumption}
We can now define the \textit{expected utility} by:
\[
    U(\tilde{x}) := \sum^S_{s=1}\pi_s u(x_s) = \E[u(\tilde{x})],
\]
where $u$ is the utility function that represents the preference relation with certainty.

\begin{theorem}[Theorem 2.1]
    If $\mathcal{R}$ satisfies the assumptions, then there exist $S$ scalars $u(x_s)$ such that:
    \[
        \forall \tilde{x}_1, \tilde{x}_2 \in \mathcal{M} : \;\; \tilde{x}_1 \mathcal{R} \tilde{x}_2 \iff U(\tilde{x}_1) \geq U(\tilde{x}_2).
    \]
\end{theorem}

\begin{proof}
    \todo{Learn this proof!}.
\end{proof}

\section{Risk Aversion}
A gamble $\tilde{x}$ is an \textit{actuarially fair} gamble if:
\[
    \sum^S_{s=1}\pi_s x_s = 0.
\]
We have the following types of agents:
\begin{itemize}
    \item \textit{Risk averse}: Agent does not accept or is indifferent to any actuarially fair gamble at all wealth levels.
    \item \textit{Risk neutral}: Agent is indifferent to any actuarially fair gamble.
    \item \textit{Risk loving}: Agent does accept any actuarially fair gamble.
\end{itemize}
The \textit{risk premium} of a gamble $\tilde{x}$ is the amount an agent is willing to pay to exchange his expected wealth for certain wealth, i.e. to avoid the gamble. That is, the amount $\rho_u(\tilde{x})$ such that:
\begin{align*}
    u\left(\E[\tilde{x}] - \rho_u(\tilde{x})\right) = \E[u(\tilde{x})] \\[10pt]
    \rho_u(\tilde{x}) := \max \{x \in \R_{\geq0} : u(\E[\tilde{x}] - x) = U(\tilde{x})\}
\end{align*}
The \textit{certainty equivalent} of a gamble $\tilde{x}$ is the amount of money an agent wants, to exchange his investment:
\[
    \text{CE}_u(\tilde{x}) := \E[\tilde{x}] - \rho_u(\tilde{x}).
\]

\begin{proposition}
    The following are equivalent:
    \begin{enumerate}[(i)]
        \item The agent is risk-averse.
        \item $u$ is concave.
        \item $\text{CE}_u(\tilde{x}) \leq \E[\tilde{x}]$.
        \item $\rho_u(\tilde{x}) \geq 0$.
    \end{enumerate}
    \begin{explanation}
        \todo{Explain!}.
    \end{explanation}
\end{proposition}
Let $\tilde{x} = x + \tilde{\epsilon}$, where $\tilde{\epsilon}$ is a random variable with mean zero and variance $\sigma^2$, so that $\E[\tilde{x}] = x$. Now we can write:
\[
    u\left(\E[\tilde{x}] - \rho_u(\tilde{x})\right) = u(x - \rho_u(\tilde{x})) = U(\tilde{x}).
\]
Using a Taylor expansion we find that $U(\tilde{x}) = \E[u(x + \tilde{\epsilon})] \approx u(x) + \frac{\sigma^2}{2}u''(x)$, and that $u(x - \rho_u(\tilde{x}))  \approx u(x) - u'(x)\rho_u(\tilde{x})$. Hence, $\rho_u(\tilde{x}) \approx \frac{1}{2} r^a_u(x)\sigma^2$, where
\[
    r^a_u(x) := -\frac{u''(x)}{u'(x)},
\]
the coefficient of \textit{absolute risk aversion}. If we analyse the multiplicative noise $\tilde{x} = x(1 + \tilde{\epsilon})$, we find the coefficient of \textit{relative risk aversion}:
\[
    r^r_u(x) := -\frac{u''(x)}{u'(x)}x.
\]

We use the following classification of risk-aversion:
\begin{itemize}
    \item \textit{DARA}: Decreasing absolute risk aversion, $x \mapsto r^a_u(x)$ is a decreasing function.
    \item \textit{CARA}: Constant absolute risk aversion, $x \mapsto r^a_u(x)$ is a constant function.
    \item \textit{IARA}: Increasing absolute risk aversion, $x \mapsto r^a_u(x)$ is an increasing function.
\end{itemize}
We say of two agents $a$ and $b$ that $a$ is more risk averse than $b$ if agent $b$ always accepts a gamble if agent $a$ does, or if for every gamble, the risk premium of agent $a$ is greater than or equal to the risk premium of agent $b$.

\lecture{4}

\begin{proposition}
    Given two increasing and strictly concave utility functions $u^a$ and $u^b$, the following conditions are equivalent:
    \begin{itemize}
        \item $r^a_{u^a}(x) \geq r^a_{u^b}(x)$ for all $x \in \R_{\geq0}$.
        \item There exists an increasing, concave function $g$ such that $u^a(x) = g(u^b(x))$ for all $x \in \R_{\geq0}$.
        \item $u^a$ is more risk averse than $u^b$, i.e. $\rho_{u^a}(x + \tilde{\epsilon}) \geq \rho_{u^b}(x +\tilde{\epsilon})$ for all $x \in \R_{\geq0}$ and for any random variable $\tilde{\epsilon}$ with mean zero.
    \end{itemize}
\end{proposition}
\begin{proof}
    \todo{Prove!}
\end{proof}

Examples of utility functions are:
\begin{itemize}
    \item Exponential utility functions: $u(x) = -e^{-\alpha x}$, where $\alpha > 0$.
    \begin{explanation}
        The coefficient of absolute risk aversion is $r^a_u(x) = \alpha$, which is constant (and positive). Therefore, the exponential utility function is CARA and the coefficient of relative risk aversion $r^r_u(x) = \alpha x$ is increasing.
    \end{explanation}
    \item Quadratic utility functions: $u(x) = x - \frac{b}{2}x^2$.
    \begin{explanation}
        The coefficient of absolute risk aversion is $r^a_u(x) = \frac{b}{1 - bx}$, which is increasing and positive. Therefore, the quadratic utility function is IARA and the coefficient of relative risk aversion $r^r_u(x) = \frac{bx}{1 - bx}$ is increasing.
    \end{explanation}
    \item Power utility functions: $u(x) = \frac{b}{b-1}x^{1 - \frac{1}{b}}$, where $b > 0$.
    \begin{explanation}
        The coefficient of absolute risk aversion is $r^a_u(x) = \frac{1}{bx}$, which is decreasing and positive. Therefore, the power utility function is DARA and the coefficient of relative risk aversion $r^r_u(x) = \frac{1}{b}$ is constant.
    \end{explanation}
    \item Logarithmic utility functions: $u(x) = \ln(bx)$, where $b > 0$.
    \begin{explanation}
        The coefficient of absolute risk aversion is $r^a_u(x) = \frac{1}{x}$, which is decreasing and positive. Therefore, the logarithmic utility function is DARA and the coefficient of relative risk aversion $r^r_u(x) = 1$ is constant.
    \end{explanation}
\end{itemize}

\section{Portfolio Problem}

An agent with initial wealth $W_0$ faces the following problem. Allocate $W_0$ among $N+1$ assets, of which $N$ are risky. The return of asset $n$ at time 1 is described by $\tilde{r}_n$, with $n = 1, \dots, N$. Asset 0 is the risk-free asset with return $r_f$. There are $S$ possible outcomes:
\[
    \begin{pmatrix}
        r_{11} & r_{12} & \dots & r_{1N} \\
        r_{21} & r_{22} & \dots & r_{2N} \\
        \vdots & \vdots & \ddots & \vdots \\
        r_{S1} & r_{S2} & \dots & r_{SN}
    \end{pmatrix}
\]
The agent invests the vector with amounts $\mathbf{w} = (w_1, \dots, w_N)$, $\mathbf{w} \in \R^{N}$. The budget constraint forces the agent to invest the rest in the risk-free asset:
\[
    W_0 - \sum_{n=0}^N w_n.
\]
The wealth at time 1 becomes $y = R\mathbf{w}^\top$. The \textit{consumption plans} $\mathbf{c}$ reachable in $\R^S$ are given by the image of the transformation $R$, written as $I(R)$. A market is complete if every consumption plan is reachable, i.e. $I(R) = \R^S$. The $N$ assets are said to be \textit{non redundant} if their returns are linearly independent, i.e. $N \leq S$ and $\text{rank}(R) = N$. The \textit{uniqueness of representation} property states that for all $\mathbf{c} \in I(R)$ there exists a unique $\mathbf{w} \in \R^N$ such that $\mathbf{c} = R\mathbf{w}^\top$. 

\lecture{5}

Assuming $R$ is full rank, the following cases are possible:
\begin{itemize}
    \item $N > S$: $I(R) = \R^S$, and $\dim\{\mathbf{w} : R\mathbf{w}^\top = \mathbf{c}\} = N-S$.
    \begin{explanation}
        The market is complete, as every possible consumption plan is reachable ($I(R) = \R^S$). Some assets are \textit{redundant}. The number of redundant assets is $N-S$, and the number of non-redundant assets is $S$.
    \end{explanation}
    \item $N = S$: $I(R) = \R^S$, and $\dim\{\mathbf{w} : R\mathbf{w}^\top = \mathbf{c}\} = 0$.
    \begin{explanation}
        The market is complete, as every possible consumption plan is reachable ($I(R) = \R^S$). There are no redundant assets, as the number of non-redundant assets is $S$.
    \end{explanation}
    \item $N < S$: $I(R) \subset \R^S$, and $\dim\{\mathbf{w} : R\mathbf{w}^\top = \mathbf{c}\} = 0$.
    \begin{explanation}
        The market is incomplete, as not every possible consumption plan is reachable ($I(R) \subset \R^S$). There are no redundant assets, as the number of non-redundant assets is $S$.
    \end{explanation}
\end{itemize}
The wealth at $t_1$ associated with the portfolio is given by:
\begin{align*}
    \tilde{W} &= \left(W_0 - \sum_{n=1}^N w_n\right)r_f + \sum_{n=1}^N w_n \tilde{r}_n \\
    &= W_0r_f + \sum_{n=1}^N w_n(\tilde{r}_n - r_f).
\end{align*} 
An agent would maximize his expected utility given an initial wealth. Hence, the agent's \textit{optimal portfolio problem} is given by:
\[
    \max_{\mathbf{w}} \E[u(\tilde{W})], \;\;\; \mathbf{w} \in \R^N.
\]
The first order necessary condition is given by $\E[u'(\tilde{W})(\tilde{r}_n - r_f)] = 0$, with $n = 1, \dots, N$. If $u'' < 0$, then there is a \textit{unique wealth value} $W$ that yield the maximum of the utility functions. This will be attained by a unique portfolio if the assets statisfy the uniqueness of representation property. The \textit{return risk premium} is defined as:
\[
    \E[\tilde{r}_n - r_f].
\]

\begin{proposition}
    Consider a strictly risk averse agent with a strictly increasing utility function. A sufficient condition such that $\mathbf{w}^* \not\in \R^N_{\geq 0}$ is that the return risk premium of all assets is non positive, and negative for at least one asset.
    \begin{explanation}
        This constitutes an arbitrage opportunity. \todo{more explanation?}
    \end{explanation}
\end{proposition}

\begin{proposition}
 A necessary and sufficient conditions so that $\mathbf{w}^* = 0$ for an agent with a strictly increasing and strictly concave utility function is that
\[
    \E[\tilde{r}_n - r_f]  = r_f, \;\;\; n = 1, \dots, N.
\]
\begin{explanation}
    \todo{Explain.}
\end{explanation}
\end{proposition}

\begin{proposition}
    Given $N$ risky assets with returns $\tilde{r}_1, \dots, \tilde{r}_N$, and a risk-free asset with return $r_f > 0$. Let $u$ be differentiable, strictly increasing and concave. Then for $n = 1, \dots, N$:
    \begin{align*}
        \E[\tilde{r}_n - r_f] &\geq 0 \iff \mathbf{w}^* = cov\left(\tilde{r}_n, u'(W_0\tilde{r}^*)\right) \leq 0, \\
        \E[\tilde{r}_n - r_f] &\leq 0 \iff \mathbf{w}^* = cov\left(\tilde{r}_n, u'(W_0\tilde{r}^*)\right) \geq 0.
    \end{align*}
    \begin{explanation}
        A risk-averse agent wants to reduce the variance of his portfolio. For assets that do this (i.e. negative correlation between asset returns and wealth), a negative risk premium is accepted. For assets that increase the variance of the portfolio, a positive risk premium is required.
    \end{explanation}
\end{proposition}

For returns with a small dispersion, the following approximation holds:
\[
    \E[\tilde{r}_n - r_f] \approx - \frac{u''\left(\E[\tilde{W}^*]\right)}{u'\left(\E[\tilde{W}^*]\right)} cov\left(\tilde{r}_n, \tilde{W}^*\right)
\]

\lecture{6}

\begin{proposition}
    If the returns $\tilde{r}_n$, with $n = 1, \dots, N$, are iid, then the optimal portfolio for a risk averse agent is given by:
    \[
        w_n = \frac{W_0}{N}.
    \]
\end{proposition}

\begin{proposition}
    If an agent is risk averse with a strictly increasing utility function, then:
    \[
        \E[\tilde{r}_n] - r_f \geq 0
    \]
\end{proposition}

Limit our portfolio problem to one risky and one risk-free asset. By $w(x)$ we denote the risky asset demand of an agent with initial wealth $x$.

\begin{proposition}
    Let $u$ be an increasing, three times differentiable and strictly concave utility function. Then:
    \begin{align*}
        r^{a'}_u(z) < 0 \;\;\; \forall z \in \R_{\geq 0} & \implies w'(x) > 0 \;\;\; \forall x \in \R_{\geq 0} \\
        r^{a'}_u(z) > 0 \;\;\; \forall z \in \R_{\geq 0} & \implies w'(x) < 0 \;\;\; \forall x \in \R_{\geq 0} \\
        r^{a'}_u(z) = 0 \;\;\; \forall z \in \R_{\geq 0} & \implies w'(x) = 0 \;\;\; \forall x \in \R_{\geq 0}
    \end{align*}
\end{proposition}

The absolute risk aversion coefficient gives information of changes in the levels of the risky asset demand, not in the percentage of wealth invested in the risky asset. For that, we use the relative risk aversion coefficient.

\begin{proposition}
    Let $u$ be an increasing, three times differentiable and strictly concave utility function. Then:
    \begin{align*}
        r^{r'}_u(z) < 0 \;\;\; \forall z \in \R_{\geq 0} & \implies \frac{dw}{dx}\frac{x}{w} < 1 \;\;\; \forall x \in \R_{\geq 0} \\
        r^{r'}_u(z) > 0 \;\;\; \forall z \in \R_{\geq 0} & \implies \frac{dw}{dx}\frac{x}{w} > 1 \;\;\; \forall x \in \R_{\geq 0} \\
        r^{r'}_u(z) = 0 \;\;\; \forall z \in \R_{\geq 0} & \implies \frac{dw}{dx}\frac{x}{w} = 1 \;\;\; \forall x \in \R_{\geq 0}
    \end{align*}
\end{proposition}

\section{Insurance, Demand and Prudence}
An \textit{Arrow} security delivers at time $t=1$ exactly one unit of wealth if an elementary event is realized, and zero otherwise. Suppose a strictly risk averse agent endowed in $t=0$ with an amount of money $W_0>0$ faces a possible monetary loss $D>0$ at time $t=1$, where the loss occurs with probability $0 < \pi < 1$. The expected utility is given by:
\[
    \pi u(W_0 - D) + (1-\pi)u(W_0).
\]
Suppose the agent can buy an Arrow security that pays one unit of wealth in case the loss occurs. The price of this unit of wealth is $q>0$. The agent must solve the following optimization problem:
\[
    \max_{q} \pi u(W_0 - D - wq + w) + (1-\pi)u(W_0 - wq), \;\;\; w \in \R_{\geq 0}.
\]
Here $w^* \geq 0$ is a solution of the problem if and only if:
\[
    \pi(1-q) u'(W_0 - D - w^*q + w^*) - q(1-\pi)u'(W_0 - w^*q) \leq 0,
\]
with equality for $w^* > 0$. The Arrow security is in fact a gamble: $[1-q, -q; \pi, 1-\pi]$. The gamble is actuarially fair for $q = \pi$. So, for $w^* > 0$ we have
\[
    u'\left(W_0 - D + w^*(1-q)\right) = u'\left(W_0 - w^*q\right).
\]
By concavity of $u$ we find $w^* = D$, so for actuarially fair price, the risk averse agent will insure themselve completely. If the gamble is not fair, then there is partial insurance in case $q > \pi$, and over insurance in case $q < \pi$.

Consider now the generic gamble $[x_1, x_2; \pi, 1-\pi]$. The ratio of the marginal utilities in two states of the world, weighted by the probability of that state, denoted $\text{SMS}(x_1, x_2)$, is called the \textit{marginal rate of substitution}. This is equal to minus the slope of the tangent line to the indifference curve at the point $(x_1, x_2)$:
\[
    \text{SMS}(x_1, x_2) = \left.-\frac{dx_2}{dx_1}\right|_{U(x) = \tilde{U}} = - \frac{\pi u'(x_1)}{(1-\pi)u'(x_2)}.
\]

\lecture{7}

\section{Stochastic Dominance}

\section{Mean-Variance Analysis}

\section{Portfolio Frontier}


\section{Portfolio Frontier with a Risk-Free Asset}

\begin{figure}[ht]
    \centering
    \incfig{epf}
    \caption{The efficient portfolio frontier.}
    \label{fig:epf}
\end{figure}

\section{CAPM}

\section{Empirical Test CAPM}

\section{Arbitrage Pricing Theory}

\section{Fundamental Theorem of Asset Pricing}


\end{document}