\usepackage[utf8]{inputenc}
\usepackage[T1]{fontenc}

\usepackage{multicol}
\usepackage{enumerate}
\usepackage{graphicx}
\usepackage[font=small,labelfont=bf]{caption}
\usepackage[dvipsnames]{xcolor}

\usepackage[a4paper]{geometry}
\geometry{
  a4paper,
  left=25mm,
  right=25mm,
  top=30mm,
  bottom=30mm
}

% Work in progress
\usepackage{tikz}
\AddToHook{shipout/foreground}{
  \begin{tikzpicture}[remember picture,overlay]
    \node[gray,rotate=30,scale=20,opacity=0.09] at (current page.center) {Draft}; 
  \end{tikzpicture}
}


% Better fonts
% \usepackage{newpxtext}
\usepackage{cmbright}
\usepackage{lmodern}

% Removes indentation and adds vspace
%   NOTE: add 1em below each theorem if disabled
\usepackage{parskip}

% Equations and mathematical symbols
\usepackage{amsmath, amssymb, amsthm, mathtools}
\usepackage{calrsfs}
\usepackage{textcomp, gensymb}

% Hyperlinks
\usepackage{hyperref}
\usepackage[noabbrev]{cleveref}

% Appendices
\usepackage{appendix}


% Code listings
\usepackage{listings}
\usepackage{transparent}
\lstset{frame=tb,
  language=Python,
  aboveskip=7mm,
  belowskip=7mm,
  columns=flexible,
  basicstyle={\footnotesize\ttfamily},
  keywordstyle={\bfseries\color{NavyBlue}},
  commentstyle=\color{Gray},
  stringstyle=\color{Green},
  breaklines=false,
  breakatwhitespace=true,
  tabsize=3,
  morecomment=[l][\color{Gray}\transparent{0.5}]{\#},
}


% Theorems
\usepackage{tikz-cd}
\usepackage{thmtools}
\usepackage[framemethod=TikZ]{mdframed}
\mdfsetup{skipabove=1em, skipbelow=0em, innertopmargin=5pt, innerbottommargin=6pt} 

\theoremstyle{definition}

\declaretheoremstyle[headfont=\bfseries\sffamily, bodyfont=\normalfont, mdframed={ nobreak,}]{thmbox}
\declaretheoremstyle[headfont=\bfseries\sffamily, bodyfont=\normalfont, numbered=no, mdframed={ nobreak, rightline=false, topline=false, bottomline=false}]{infobox}

\declaretheorem[style=thmbox, name=Definition]{definition}
\declaretheorem[sibling=definition, style=thmbox, name=Corollary]{corollary}
\declaretheorem[sibling=definition, style=thmbox, name=Proposition]{proposition}
\declaretheorem[sibling=definition, style=thmbox, name=Theorem]{theorem}
\declaretheorem[sibling=definition, style=thmbox, name=Lemma]{lemma}

\declaretheorem[style=infobox, name=Assumption]{assumption}
\declaretheorem[style=infobox, name=Explanation]{explanation}
\declaretheorem[style=infobox, name=Exercise]{ex}
\declaretheorem[style=infobox, name=Example]{eg}
\declaretheorem[style=infobox, name=Remark]{remark}
\declaretheorem[style=infobox, name=Note]{note}

\declaretheoremstyle[headfont=\bfseries\sffamily, bodyfont=\normalfont, numbered=no, mdframed={} ]{thmsolutionbox}
\declaretheorem[numbered=no, style=thmsolutionbox, name=Solution]{solution}


\usepackage{import}
\usepackage{xifthen}
\usepackage{pdfpages}
\usepackage{transparent}

\newcommand{\incfig}[2][0.8]{%
    \def\svgwidth{#1\textwidth}
    \import{./figures/}{#2.pdf_tex}
}


\makeatletter

% Set space around theorems
\def\thm@space@setup{%
  \thm@preskip=.5\topsep \thm@postskip=0pt
}

\def\@lecture{}%
\newcommand{\lecture}[1]{
    \def\@lecture{Lecture #1}
}

% These are the fancy headers
\usepackage{fancyhdr}
\usepackage{emptypage}
\pagestyle{fancy}
\setlength{\headheight}{13pt}

% LE: left even
% RO: right odd
% CE, CO: center even, center odd
% My name for when I print my lecture notes to use for an open book exam.
% \fancyhead[LE,RO]{Auke Schaap}

\fancyhead[RO,LE]{\@lecture} % Right odd,  Left even
\fancyhead[RE,LO]{}          % Right even, Left odd

\fancyfoot[RO,LE]{\thepage}  % Right odd,  Left even
\fancyfoot[RE,LO]{}          % Right even, Left odd
\fancyfoot[C]{\leftmark}     % Center

\makeatother




%%%%%%%%%%%%%%%%%
% Custom macros %
%%%%%%%%%%%%%%%%%

\newcommand\N{\ensuremath{\mathbb{N}}}
\newcommand\R{\ensuremath{\mathbb{R}}}
\newcommand\E{\ensuremath{\mathbb{E}}}


% (*) above a symbol
\newcommand\mygt{\stackrel{\mathclap{\scriptsize\mbox{$(\ast)$}}}{>}}

\newif\ifshowcomments
\showcommentstrue
% \showcommentsfalse
\ifshowcomments
\newcommand{\mynote}[2]{\fbox{\bfseries\sffamily\scriptsize{#1}}{\small$\blacktriangleright$\textsf{\emph{#2}}$\blacktriangleleft$}}
\newcommand{\citehere}[0]{\textcolor{red}{\fbox{\bfseries\sffamily\scriptsize{CITATION}}}}
\else
\newcommand{\mynote}[2]{}
\fi
\newcommand{\todo}[1]{\textcolor{blue}{\mynote{To do}{#1}}}